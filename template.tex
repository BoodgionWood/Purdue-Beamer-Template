% For format, please refer to TsingHua template at https://github.com/tuna/THU-Beamer-Theme
% For Purdue content and color, please refer to https://github.com/abarbu/purdue-beamer

\documentclass{beamer}
\usepackage{hyperref}
%\usepackage[numbers]{natbib}
\usepackage{natbib}
\bibliographystyle{plainnat}
\usepackage[T1]{fontenc}
\usepackage{latexsym,amsmath,xcolor,multicol,booktabs,calligra}
\usepackage{graphicx,listings,stackengine}

\author{Haoyun Yin}
\title{Wasserstein Coreset via Sinkhorn Loss}
\subtitle{JSM 2024}
\institute{Department of Statistics, Purdue University}
\date{\today}
\usepackage{Purdue}

% defs
\def\cmd#1{\texttt{\color{red}\footnotesize $\backslash$#1}}
\def\env#1{\texttt{\color{blue}\footnotesize #1}}
\definecolor{deepblue}{rgb}{0,0,0.5}
\definecolor{deepred}{rgb}{0.6,0,0}
\definecolor{deepgreen}{rgb}{0,0.5,0}
\definecolor{halfgray}{gray}{0.55}

\lstset{
    basicstyle=\ttfamily\small,
    keywordstyle=\bfseries\color{deepblue},
    emphstyle=\ttfamily\color{deepred},    % Custom highlighting style
    stringstyle=\color{deepgreen},
    numbers=left,
    numberstyle=\small\color{halfgray},
    rulesepcolor=\color{red!20!green!20!blue!20},
    frame=shadowbox,
}

\begin{document}
\begin{frame}
    \titlepage
    \begin{figure}[htpb]
        \begin{center}
            \includegraphics[width=0.3\linewidth]{pic/PU_signature_eps}
        \end{center}
    \end{figure}
\end{frame}

\begin{frame}
\tableofcontents[sectionstyle=show,
subsectionstyle=show/shaded/hide,
subsubsectionstyle=show/shaded/hide]
\end{frame}

\section{Introduction}

\begin{frame}
\frametitle{Introduction}
This presentation discusses the challenges in data science, including missing, noisy, and imbalanced data. We will explore the concept of Wasserstein Coreset via Sinkhorn Loss.

\end{frame}

\section{Backgrounds}
\subsection{Wasserstein Distance \& Sinkhorn Loss}
% without [<+->], all items are shown at once
\begin{frame}[<+->]
\frametitle{Related Work}
\begin{itemize}
\item The Wasserstein distance is discussed in \citep{villani2009optimal}.
\item The Sinkhorn loss and algorithm is introduced in \citep{cuturi2013sinkhorn}.
\item The relationship between Wasserstein distance, MMD and Sinkhorn loss is discussed in \citep{feydy2019interpolating}.
\end{itemize}
\end{frame}

\subsection{Coresets}
\begin{frame}[<+->]
\frametitle{Coresets}
\begin{itemize}
    \item Coreset is a small weighted subset of the data that approximates the original data.
    \item The concept of Wasserstein Coreset is discussed in \citep{ClaiciSebastian2018WMC}.
\end{itemize}
\end{frame}

\section{Methodology}
\begin{frame}[<+->]
\frametitle{Methodology}
\begin{itemize}
    \item We propose a new method to construct Wasserstein Coreset via Sinkhorn Loss.
    \item The method is suitable for high-dimensional data.
    \item The method is efficient and stable.
\end{itemize}
\end{frame}

\begin{frame}[allowframebreaks]
\frametitle{References}
\tiny\bibliography{references}
\end{frame}

\begin{frame}
    \begin{center}
        {\Huge\calligra Thanks!}
    \end{center}
\end{frame}

\end{document}
